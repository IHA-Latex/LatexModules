%---------------------------------------------------------------------------%
%------------------------ Text configuration -------------------------------%
%---------------------------------------------------------------------------%
\usepackage[english, danish]{babel} % Language package

\usepackage[utf8]{inputenc} 		% Input encoding - Depending on editor
\usepackage{lmodern} 				% Modern LaTeX font
\usepackage[T1]{fontenc} 			% Hyphenation
\usepackage{textcomp} 				% Symbols package
\usepackage{gensymb} 				% Provides generic commands \degree, \celsius, \perthousand, \micro and \ohm
\usepackage{ragged2e,anyfontsize} 	% Adjust text
\usepackage{relsize}				% Relativ font size change


\setlength\parindent{0pt} 			% No indent
\setlength\parskip{12pt} 			% More than a single line break will give ONE linebreak.



%---------------------------------------------------------------------------%
%-------------------------- Tables and lists -------------------------------%
%---------------------------------------------------------------------------%
\usepackage{tabu}					% Tabu environment
\usepackage{multirow} 				% To merge rows
\usepackage{longtable}				% Show table over multiple pages.
\usepackage{hhline} 				% Horizonal lines with 'multirow'
% \usepackage[shortlabels]{enumitem}	% Options for enumerate and itemize
\usepackage{color, colortbl}		% Colors in rows and cells

\definecolor{gr}{gray}{0.9}

% Example of usage:
%--------------------------------------------------
% \begin{table}
% 	\begin{tabular}{ l c r }
% 	  1 & 2 & 3 \\
% 	  \rowcolor{gr} % This row will be gray
% 	  4 & 5 & 6 \\
% 	  7 & 8 & 9 \\
% 	  \rowcolor{gr} % This row will be gray
% 	  10 & 11 & 12 \\
% 	\end{tabular}
% \end{table}
%--------------------------------------------------



%---------------------------------------------------------------------------%
%-------------------- UML diagrams directly in Latex -----------------------%
%---------------------------------------------------------------------------%
\usepackage{uml} 					% UML diagrams
%\usepackage{MetaUML}				% UML diagrams
\usepackage{tikz} 					% Graphical tool
\usepackage[underline=true]{pgf-umlsd}
\usetikzlibrary{arrows,shadows} 	% for pgf-umlsd


%---------------------------------------------------------------------------%
%---------------------------- Subfiles -------------------------------------%
%---------------------------------------------------------------------------%
% Use multiple files in the same document
\usepackage{subfiles}

% Begin each document as follow:
%--------------------------------------------------
%\documentclass[Name-Of-MasterFile]{subfiles}
%\begin{document}
%Content
%\end{document}
%--------------------------------------------------

% In the master file include the following:
%--------------------------------------------------
%\subfile{NameOfSubfile}
%--------------------------------------------------


%---------------------------------------------------------------------------%
%------------------------ Link and references ------------------------------%
%---------------------------------------------------------------------------%
%http://mirrors.dotsrc.org/ctan/macros/latex/required/tools/varioref.pdf
\usepackage{color}	 		% Needed for color
\usepackage{varioref} 		% Smarter references
\usepackage{makeidx}		% Package for indexing word.
\usepackage{hyperref}
\hypersetup{ 
	colorlinks	= true, 	% false: boxed links; true: colored links
    urlcolor	= blue,		% Color of external links
    linkcolor	= black, 	% Color of page numbers
    citecolor	= blue,		% Citation color
}

% Example of usage:
%--------------------------------
% \url{http://www.wikibooks.org}
% \href{http://www.wikibooks.org}{Wikibooks home}
%--------------------------------

%---------------------------------------------------------------------------%
%------------------------ Cross references to documents --------------------%
%---------------------------------------------------------------------------%
\usepackage{xr}			% Cross references to other .aux files than the current file
%\usepackage{nameref,zref-xr}
%\zxrsetup{toltxlabel}
\newcommand{\extref}[1]{\autoref{doc-#1} \autopageref{doc-#1}}

% Examples of usage:
%-------------------------
% \externaldocument{Path/FileName}
% \begin{document}
% \ref{figure1}
% \end{document}
%-------------------------

%---------------------------------------------------------------------------%
%------------------------ FixMe options ------------------------------------%
%---------------------------------------------------------------------------%
% The FixMe package shows allows small notes in your files.
% You will see them as a footnote and shown in a list.

% Using 'final' instead of 'draft' will hide the notes and the list.
\usepackage[footnote,draft,danish,silent,nomargin]{fixme}

% Show a list of all \fxnote{} and \fixme{} is done with \listoffixmes

% Example of usage:
%--------------------------------------------------
% This is some text. \fxnote{Remember to add more.}
%--------------------------------------------------


%---------------------------------------------------------------------------%
%------------------------ Generel options ----------------------------------%
%---------------------------------------------------------------------------%

\usepackage{fix-cm} 		% Fix for cm
\usepackage{lipsum} 		% Debugging text: \lipsum[1-2]
\usepackage{epstopdf} 		% EMF to PDF conversion
\usepackage{fixltx2e} 		% Fixes errors in the LaTeX-kernel
\usepackage{epstopdf}		% Convert EPS to PDF files
\usepackage{etex}			% Typesetting to replace TeX3



%---------------------------------------------------------------------------%
%------------------------ Microtype package --------------------------------%
%---------------------------------------------------------------------------%
% Small adjustments to the font
% See http://www.khirevich.com/latex/microtype/ for explanation

\usepackage[activate={true,nocompatibility},final,tracking=true,kerning=true,spacing=true,factor=1100,stretch=10,shrink=10]{microtype}


\SetExtraKerning[unit=space]
    {encoding={*}, family={bch}, series={*}, size={footnotesize,small,normalsize}}
    {\textendash={400,400}, 			% en-dash, add more space around it
     "28={ ,150}, 						% left bracket, add space from right
     "29={150, }, 						% right bracket, add space from left
     \textquotedblleft={ ,150}, 		% left quotation mark, space from right
     \textquotedblright={150, }} 		% right quotation mark, space from left

\SetTracking{encoding={*}, shape=sc}{40}
\microtypecontext{spacing=nonfrench}

\let\newTOC\tableofcontents
\renewcommand{\tableofcontents}
{
	\microtypesetup{protrusion=false} 	% disables protrusion locally in the document
	\newTOC			 					% prints Table of Contents
	\microtypesetup{protrusion=true} 	% enables protrusion
}


%---------------------------------------------------------------------------%
%------------------------ Figures ------------------------------------------%
%---------------------------------------------------------------------------%
\usepackage{caption}				% Set a caption around the figure
\usepackage{float}					% Let location 'H' be an option
\usepackage{graphicx}				% Additional figure options
\usepackage{subfig}					% Captions for multiple figures
\usepackage{wrapfig}				% Text around the figure
\usepackage{rotating}				% Rotation of tables and figures
\usepackage[section]{placeins}		% Figure appear before a new \section{}

% Remove complains from Memoir regarding the caption and subcaption package.
\makeatletter
\newcommand\nocaptioninlist{\renewcommand\ext@figure{ll}}
\makeatother

% Subcaption
\captionsetup[figure]{labelfont={bf,small},textfont={it,small}}
\captionsetup[subfloat]{labelfont={bf,small},textfont={it,small},
subrefformat=parens} 

% Macro for referring to a subfigure
\newcommand{\myfigref}[2]{~\ref{#1}.\subref{#2}}

% Paths the pictures can be located
% Path is relative to the Master document
\graphicspath{
	{../Figures/}
	{Figures/}
}

% Example of usage:
%--------------------------------------------------
%\begin{figure}[hbtp]
%\centering
%\includegraphics[width =0.9 \textwidth]{filename-for-png}
%\caption{Titel}
%\label{fig:referenceNavn}
%\end{figure}
%--------------------------------------------------


% Example of figure side-by-side
%--------------------------------------------------
% \begin{figure}[ht]
% \centering
% \subfloat[My first picture]{
% 	\label{fig:mdleft}{
% 		\includegraphics[width=0.4\textwidth]{my figure}}}
% \hfill
% \subfloat[My second picture]{
% 	\label{fig:mdright}{
% 		\includegraphics[width=0.4\textwidth]{my figure}}}

% \caption{My two big pictures}
% \label{fig:subfigures}
% \end{figure}
%--------------------------------------------------


%--------------------------------------------------------------------------%
%------------------------ Defining of colors ------------------------------%
%--------------------------------------------------------------------------%
\usepackage{xcolor} 	% Color package

\definecolor{ase_blue}	{RGB}{10,	55,	 136}
\definecolor{darkviolet}{rgb}{0.58, 0.0, 0.83}
\definecolor{darkgreen}	{rgb}{0.0, 	0.2, 0.13}
\definecolor{darkblue}	{rgb}{0.0, 	0.0, 0.55}
\definecolor{darkgray}	{rgb}{0.66, 0.66,0.66}
\definecolor{webblue}	{rgb}{0.0,	0.05,0.45}
\definecolor{MyDarkBlue}{rgb}{0,	0.08,0.45}
\definecolor{webred}	{rgb}{0.75,	0,	 0}

%--------------------------------------------------------------------------%
%------------------------ Defining of Bibliography ------------------------%
%--------------------------------------------------------------------------%
%\bibpunct[,]{[}{]}{;}{a}{,}{,} 	% For Natbib
\usepackage{cite}			% Citation from bibliography
\bibliographystyle{ieeetr}	% Style of bibliography - IEEE's standard



%---------------------------------------------------------------------------%
%------------------------ Margin setup -------------------------------------%
%---------------------------------------------------------------------------%

\setlrmarginsandblock{3.5cm}{2.5cm}{*}		% Left and right margin
\setulmarginsandblock{3cm}{*}{1.2}			% Upper and lower margin
\checkandfixthelayout[nearest]				% Applies the margins


%--------------------------------------------------------------------------%
%------------------------- PAGESTYLE - PROPERTIES -------------------------%
%--------------------------------------------------------------------------%
\usepackage{lastpage}		% Will show the last page number

\renewcommand{\chaptermark}[1]{\markboth{\MakeUppercase{#1}}{}}

% Text at buttom of page: "CurrectPage / LastPage"
\newcommand{\footerText}{\thepage\xspace /\pageref{LastPage}}

\makepagestyle{ase_report}								% New style
\makeoddhead{ase_report}{}{\small\sffamily\leftmark}{}	% Header text, odd pages
\makeoddfoot{ase_report}{}{\footerText}{}				% Footer text, odd pages
\makeevenhead{ase_report}{}{\small\sffamily\leftmark}{}	% Header text, even pages
\makeevenfoot{ase_report}{}{\footerText}{} 				% Footer text, even pages

\makeatletter							% Create speciel macro
\makepsmarks{ase_report}{%				% For normal pages
  \renewcommand\chaptermark[1]{%		% Rewrite \chaptermark with 1 option
    \markboth{%							% Left and right mark
      \ifnum \value{secnumdepth} > 1 	% If section depth > 1 (subsection and below)
	      \if@mainmatter 				% If in the main pages 			
	      	\@chapapp\ \thechapter. \ 	% ... Chapter to appendix?
	      \fi
      \fi
      ##1}{}}%
  \renewcommand\tocmark{\markboth{\contentsname}{\contentsname}}%
  \renewcommand\lofmark{\markboth{\listfigurename}{\listfigurename}}%
  \renewcommand\lotmark{\markboth{\listtablename}{\listtablename}}%
  \renewcommand\bibmark{\markboth{\bibname}{\bibname}}%
  \renewcommand\indexmark{\markboth{\indexname}{\indexname}}%
  \renewcommand\sectionmark[1]{\markright{##1}}%
  \renewcommand\subsectionmark[1]{\markright{##1}}%
  \renewcommand\subsubsectionmark[1]{\markright{##1}}%
  % \renewcommand{\@evenfoot}{%						% Set footer on normal, even  pages
  % 	\normalsize\slshape \today\hfil \upshape %	% Left footer text
  % 	\footerText}								% Right footer text
  % \renewcommand{\@oddfoot}{\@evenfoot}			% Set odd pages footer equal even
}
\makeatother							% End speciel macro

\copypagestyle{plain}{ase_report}		% Copy style
\makeoddhead{plain}{}{}{}				% Remove header text
\makeevenhead{plain}{}{}{}				% Remove header text

\aliaspagestyle{chapter}{plain}			% Create alias of style 'plain' for 'chapter'

\pagestyle{ase_report} 					% Apply style to document

%--------------------------------------------------------------------------%
%--------------------- HEADING - SECTION ----------------------------------%
%--------------------------------------------------------------------------%

\newcommand{\ruledsec}[1]{%
  \Large\bfseries\sffamily\raggedright #1
  \color{ase_blue}\rule[15pt]{\textwidth}{1.0pt}} % Section with ruler
\setsecheadstyle{\ruledsec} % Define section head style

\setfloatlocations{figure}{htp}
\setfloatlocations{table}{htp}

%--------------------------------------------------------------------------%
%--------------------- HEADING - SUBs-SECTION -----------------------------%
%--------------------------------------------------------------------------%

\addtocounter{secnumdepth}{2} % Depth numbering

%\setsubsecheadstyle{\large\bfseries\sffamily\raggedright}
% \setsubsubsecheadstyle{\Large\bfseries\sffamily\raggedright}

\setsechook{\hangsecnum} % Hang the section number in margin
%\setsubsechook{\defaultsecnum} % Don't do this on the subsections
%\setsubsubsechook{\defaultsecnum}
\setaftersecskip{5pt} % Default skip between the section and text
\chapterprecistoc{Text in TOC} %Text Under TOC Headline

\addto\captionsenglish{
  %\renewcommand*{\cftchaptername}{Chapter{\space}}
  \renewcommand*{\cftfigurename}{Fig.{\space}}
  \renewcommand*{\contentsname}{Table of Contents}
  \renewcommand*{\abstractname}{Abstract}
  \renewcommand*{\listfigurename}{List{\space}of{\space}Figures}
  \renewcommand*{\listtablename}{List{\space}of{\space}Tables}
  \renewcommand*{\appendixtocname}{Appendices}
  \renewcommand*{\appendixpagename}{Appendices}
}

\addto\captionsdanish{
  %\renewcommand*{\cftchaptername}{Kapitel\space}
  \renewcommand*{\cftfigurename}{Fig.\space}
  \renewcommand*{\abstractname}{Resumé}
  \renewcommand*{\contentsname}{Indholdsfortegnelse}
  \renewcommand*{\listfigurename}{Liste{\space}af{\space}Figurer}
  \renewcommand*{\listtablename}{Liste{\space}af{\space}Tabeller}
  \renewcommand*{\appendixtocname}{Appendikser}
  \renewcommand*{\appendixpagename}{Appendikser}
}
%------------------------%
%---- CHAPTER STYLE -----%
%------------------------%
\makechapterstyle{ase_chapterstyle}{
  \setlength{\beforechapskip}{10pt}
  \setlength{\afterchapskip}{0.5cm}
  \renewcommand*{\printchaptername}{}
  \renewcommand*{\chapnumfont}{\normalfont\sffamily\bfseries\fontsize{60}{0}\selectfont}
  \renewcommand*{\printchapternum}{
    \flushright
    \begin{tikzpicture}
      \draw[fill,color=ase_blue] (0,0) rectangle (2cm,2cm);
      \draw[color=white] (1cm,1cm) node { \chapnumfont\thechapter };
    \end{tikzpicture}
  }
  \renewcommand*{\chaptitlefont}{\normalfont\sffamily\Huge\bfseries\color{ase_blue}}
  \renewcommand*{\printchaptertitle}[1]{%
    \raggedright\chaptitlefont\parbox[t]{\textwidth}{\raggedright##1}}
}

\chapterstyle{ase_chapterstyle}


%--------------------------------------------------------------------------%	
%------------------------- FRONTPAGE - PROPERTIES -------------------------%
%--------------------------------------------------------------------------%

\usepackage{soul} % Letterspace package
\sodef\an{}{0.05em}{.5em plus.6em}{1em plus.1em minus.1em}
\newcommand\stext[1]{\an{\scshape#1}}
\newcommand{\logoHuge}{\fontsize{0.55cm}{0.8cm}\selectfont}
\newcommand{\SuperHuge}{\fontsize{1.2cm}{1.8cm}\selectfont}


%--------------------------------------------------------------------------%
%------------------------- TOC - PROPERTIES -------------------------------%
%--------------------------------------------------------------------------%

\raggedbottomsectiontrue 		% The page may not be strected on page breaks
\setsecnumdepth{subsubsection} 	% Set section depth in the TOC
\maxsecnumdepth{subsubsection} 	% Max of section depth in the TOC
\settocdepth{subsection} 		% Up to and including subsection

\setlength{\cftbeforechapterskip}{1.0em plus 0.1em minus 0.1em} % Space from chapters


%---------------------------------------------------------------------------%
%------------------------------ MATH Libraries -----------------------------%
%---------------------------------------------------------------------------%
% http://ctan.mirrorcatalogs.com/macros/latex/contrib/siunitx/siunitx.pdf
\usepackage{ulem} % Underlining of words with various line types

% Example of use:
% \uuline{result} for double lines under a result

\usepackage{amsmath}	% Allows math mode
\usepackage{amsfonts}	% Speciel character when writing in math mode
\usepackage{amssymb}	% Speciel symbols for math mode
\usepackage{mathtools}	% Extra features to the mathematical notation
\usepackage[binary-units=true]{siunitx}	% SI units. See link above


\DeclarePairedDelimiter\ceil{\lceil}{\rceil}
\DeclarePairedDelimiter\floor{\lfloor}{\rfloor}

% Example of usage:
%-------------------------------
% $x = \ceil{2.4} = 3$
% $x = \floor{2.4} = 2$
%-------------------------------



%---------------------------------------------------------------------------%
%------------------------------ LstListing ---------------------------------%
%---------------------------------------------------------------------------%

% Følgende er til koder.
%----------------------------------------------------------
%\begin{lstlisting}[caption=Overskrift på boks, style=Code-C++, label=lst:referenceLabel]
%public void hello(){}
%\end{lstlisting}
%----------------------------------------------------------

% Exstra space
\usepackage{xspace}
% Name on the box followed by \xspace will give the name, a space, and the number.
% Edit title of \codeTitle, if another title is wanted.
\newcommand{\codeTitle}{Code snippet\xspace}

% Packages for lstlisting
\usepackage{listings}
\usepackage{color}
\usepackage{textcomp}
\definecolor{lbcolor}{rgb}{0.9,0.9,0.9}

\renewcommand{\lstlistingname}{\codeTitle} % Set the new title of the box

\lstdefinestyle{Code}
{
%	aboveskip		= {1.5\baselineskip},
 	backgroundcolor = \color{lbcolor},
	basicstyle		= \small\ttfamily,
	breakatwhitespace= false,
	breaklines		= true,
  	columns			= fixed,
	commentstyle	= \color{darkgreen},
	emphstyle		= \color{red}\bfseries,
  	extendedchars	= true,
	frame 			= lines,
	framexrightmargin	= 0pt, %6pt
	identifierstyle	= \ttfamily,
	keywordstyle	= \color{darkviolet}\bfseries,
	lineskip		= 1pt,
	literate 		= {~}{$\sim$}1 {æ}{\ae}1 {ø}{\oe}1 {å}{\aa}1 {Æ}{\AE}1 {Ø}{\OE}1 
{Å}{\AE}1,
	morecomment		= [s][\color{lightblue}]{/**}{*/},
	numbers			= left, % Want numbers? If not, outcomment this line
	numbersep		= 6pt,
	numberstyle		= \footnotesize,
	prebreak 		= \raisebox{0ex}[0ex][0ex]{\ensuremath{\hookleftarrow}},
	showstringspaces= false,
	stepnumber		= 2,
	stringstyle		= \color{darkblue},
	tabsize			= 2,
%	upquote			= true,
}

%Width needed for the frame - this must be 6 pt
\usepackage{calc}


% Styles for different code languages.
\usepackage{caption}
\DeclareCaptionFont{white}{\color{white}}
\DeclareCaptionFormat{listing}%
{\colorbox[cmyk]{0.43, 0.35, 0.35,0.35}{\parbox{\textwidth - \marginparsep}{\hspace{5pt}#1#2#3}}}

\captionsetup[lstlisting]
{
	format			= listing,
	labelfont		= white,
	textfont		= white, 
	singlelinecheck	= false, 
	width			= \textwidth - \marginparsep,
	margin			= 0pt, 
	font			= {bf,footnotesize}
}

\lstdefinestyle{Code-C} 	{language = C, 			style=Code}
\lstdefinestyle{Code-Java} 	{language = Java, 		style=Code}
\lstdefinestyle{Code-C++} 	{language = [Visual]C++,style=Code}
\lstdefinestyle{Code-VHDL} 	{language = VHDL, 		style=Code}
\lstdefinestyle{Code-Bash} 	{language = Bash,  		style=Code}
\lstdefinestyle{Code-Matlab}{language = Matlab,		style=Code}
\lstdefinestyle{Code-Prolog}{language = Prolog,		style=Code}